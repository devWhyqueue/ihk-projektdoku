% !TEX root = ../Projektdokumentation.tex
\section{Einleitung}
\label{sec:Einleitung}
Diese Projektdokumentation zu der Webanwendung \projektName für die \ac{KVWL}, welche im Rahmen der IHK-Abschlussprüfung Sommer 2019 erstellt wurde, beschreibt umfassend \ua die Thematik, notwendige Prozessschritte und erzielte Ergebnisse des Projekts. Zum besseren Verständnis erhält sie sowohl ergänzende, praxisbezogene Unterlagen und Dokumente (wie Tabellen, Grafiken, Screenshots und Quellcode) als auch eine Kundendokumentation, die Bestandteil des Anhangs ist.

\subsection{Projektumfeld} 
\label{sec:Projektumfeld}
Die Webapplikation wurde im Rahmen eines internen Projekts bei der \ac{KVWL} in Dortmund umgesetzt. Sie wurde während des Projektzeitraums vollständig auf Basis der Wünsche des Auftraggebers neu entwickelt und soll in dessen Umgebung/Umfeld nach Projektende produktiv eingesetzt werden.

\subsubsection{Unternehmen} 
\label{sec:Unternehmen}
Laut offzieller Unternehmenswebsite\footnote{\cite{KVWL}} vertritt die \ac{KVWL}
\begin{displayquote}
	die Interessen von über 15.000 niedergelassenen Vertragsärzten und Ärztlichen und Psychologischen Psychotherapeuten sowie Kinder- und Jugendlichenpsychotherapeuten in Westfalen-Lippe. Mit rund 2000 Mitarbeitern, von denen ca. die Hälfte im Notfalldienst überwiegend geringfügig beschäftigt ist, ist die \ac{KVWL} eine der größten Kassenärztlichen Vereinigungen. Alle zur vertragsärztlichen Versorgung zugelassenen Ärzte, Ärztlichen und Psychologischen Psychotherapeuten, sowie Kinder- und Jugendlichenpsychotherapeuten sind Pflichtmitglieder der \ac{KVWL}. Zu den zentralen Aufgaben der \ac{KVWL} zählt die Sicherstellung der ambulanten ärztlichen Versorgung. [...]\\
	Als öffentlich-rechtliche Körperschaft handelt die Kassenärztliche Vereinigung in Vertretung ihrer Mitglieder mit den Verbänden der Krankenkassen die Gesamtvergütung für die ärztlichen Leistungen aus. Dazu schließt sie entsprechende Verträge ab. Als genossenschaftlicher Zusammenschluss garantiert die \ac{KVWL} somit, dass die wirtschaftlichen Interessen ihrer Mitglieder gegenüber den Krankenkassen gewahrt werden.
\end{displayquote}
Darüber hinaus trägt sie dafür Sorge, dass eine ausreichende ärztliche Versorgung sichergestellt ist (Sicherstellungsauftrag) und die Abrechnungen ihrer Mitglieder sachlich-rechnerisch korrekt sind (Gewährleistungsauftrag), wofür unterstützend auch eine Stelle zur Bekämpfung rechtwidriger Finanzmittelnutzung im Gesundheitswesen eingerichtet wurde.\footnote{\Vgl \cite{KVWL}}

\subsubsection{Auftraggeber} 
\label{sec:Auftraggeber}

\begin{itemize}
	\item Wer ist Auftraggeber/Kunde des Projekts?
\end{itemize}


\subsection{Projektziel} 
\label{sec:Projektziel}
\begin{itemize}
	\item Worum geht es eigentlich?
	\item Was soll erreicht werden?
\end{itemize}


\subsection{Projektbegründung} 
\label{sec:Projektbegruendung}
\begin{itemize}
	\item Warum ist das Projekt sinnvoll (\zB Kosten- oder Zeitersparnis, weniger Fehler)?
	\item Was ist die Motivation hinter dem Projekt?
\end{itemize}


\subsection{Projektschnittstellen} 
\label{sec:Projektschnittstellen}
\begin{itemize}
	\item Mit welchen anderen Systemen interagiert die Anwendung (technische Schnittstellen)?
	\item Wer genehmigt das Projekt \bzw stellt Mittel zur Verfügung? 
	\item Wer sind die Benutzer der Anwendung?
	\item Wem muss das Ergebnis präsentiert werden?
\end{itemize}


\subsection{Projektabgrenzung} 
\label{sec:Projektabgrenzung}
\begin{itemize}
	\item Was ist explizit nicht Teil des Projekts (\insb bei Teilprojekten)?
\end{itemize}
