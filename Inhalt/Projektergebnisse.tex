% !TEX root = ../Projektdokumentation.tex
\section{Projektergebnisse}\label{sec:Projektergebnisse}

\subsection{Zeitplanung}\label{sec:Zeitplanung}
Das Projekt wurde fristgerecht zum Ende des Projektzeitraums erfolgreich abgeschlossen. Wie in Tabelle~\ref{tab:Vergleich} zu erkennen ist, konnte die Zeitplanung bis auf wenige Ausnahmen eingehalten werden.
\tabelle{Zeitplanung}{tab:Vergleich}{Zeitnachher.tex}\\
Während bei der Wirtschaftlichkeitsanalyse und der Implementierung aufgrund vorhandener Kenntnisse Zeit eingespart werden konnte, musste durch die Verzögerung bei der Einarbeitung ins Projektumfeld und der Erstellung der Dokumentation die Pufferzeit vollständig aufgebraucht werden. Dies lag zum einen an der Komplexität der bestehenden Datenstrukturen der ReportServer-Datenbank und zum anderen an dem Umfang der Dokumentationsartefakte. Da bei der Planung der Meilensteine jedoch ausreichende Pufferzeiten berücksichtigt wurden, kam es bei den Terminen zu keinen Verzögerungen.

\subsection{Soll-/Ist-Vergleich}\label{sec:SollIstVergleich}
Das in der Einleitung unter Abschnitt \ref{sec:Projektziel} definierte Projektziel wurde vollständig erreicht. Die Applikation bietet einen guten Einstieg zur Erfassung von Statistiken mit dem Ziel der Identifikation ungenutzter Berichte und Datenquellen. Dies wurde vom Auftraggeber nach Abnahme bestätigt, welcher sich nicht nur mit der Funktionalität, sondern auch mit der Gestaltung der Benutzeroberfläche hochzufrieden zeigte.

\subsection{Retrospektive}\label{sec:Retrospektive}
Nach Beendigung des Projekts kann festgehalten werden, dass eine gründliche Projektplanung einen sehr hohen Stellenwert besitzt. Sie ist besonders dann unerlässlich, wenn der zur Verfügung stehende Zeitraum knapp bemessen ist. Die Planung hilft zentrale Ziele nicht aus den Augen zu verlieren und den Arbeitsaufwand auf wesentliche Aspekte zu fokussieren. \\
Mit dem zur Prozessplanung eingesetzten Wasserfallmodell konnten sehr gute Erfahrungen gemacht werden. Dadurch, dass eine ausführliche Anforderungsanalyse mit anschließendem Systementwurf erfolgte, hatte das \ac{CFT} \teamName bereits in den Anfängen des Projekts ein detailliertes Bild der entstehenden Anwendung. So konnten noch vor der Implementierung Unstimmigkeiten und Anforderungsabweichungen vermieden werden. \\
Zudem hat es sich als großen Vorteil erwiesen, parallel mit einem größeren Benutzerkreis zu kommunizieren. In vergangenen Projekten kam es häufig zu Terminengpässen, da der Kunde in wichtigen Projektphasen nicht durchgehend erreichbar war. \\
Aus technischer Perspektive im Hinblick auf das noch für den Prüfling unbekannte Framework Querydsl lässt sich festhalten, dass diese Wahl eine gute Entscheidung war. SQL-Abfragen ließen sich mühelos in der Syntax von Java abbilden und garantierten so Typsicherheit bei den Datenzugriffen. Anzumerken ist jedoch, dass Querydsl zurzeit nicht mehr weiterentwickelt wird und folglich die Kompatibilität mit neueren Java- und Hibernate-Versionen nicht mehr sichergestellt ist.\footnote{\cite{github:qdsl}}

\subsection{Ausblick}\label{sec:Ausblick}
Obwohl das Projektergebnis den definierten Anforderungen und Projektziel vollständig genügt, bietet es hohes Erweiterungspotential. Es lässt sich bereits aus den Mock-Ups (\Anhang{app:Mockup}) entnehmen, dass noch eine Reihe weiterer Statistiken erhoben werden können. Interessant für die tägliche Arbeit des \ac{CFT} \teamName wäre \zB alle von einer Datenbanktabelle abhängigen Berichte einsehen zu können. Bei geplanten Tabellenmodifikationen könnten unerwartete Nebeneffekte auf abhängige Berichte schneller analysiert und beseitigt werden.  Außerdem können die Statistiken noch anschaulicher (\zB in Diagrammen, Grafiken) visualisiert werden, um eine höhere Benutzerfreundlichkeit zu erreichen.