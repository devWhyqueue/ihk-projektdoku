% !TEX root = Projektdokumentation.tex

% Es werden nur die Abkürzungen aufgelistet, die mit \ac definiert und auch benutzt wurden. 
%
% \acro{VERSIS}{Versicherungsinformationssystem\acroextra{ (Bestandsführungssystem)}}
% Ergibt in der Liste: VERSIS Versicherungsinformationssystem (Bestandsführungssystem)
% Im Text aber: \ac{VERSIS} -> Versicherungsinformationssystem (VERSIS)

% Hinweis: allgemein bekannte Abkürzungen wie z.B. bzw. u.a. müssen nicht ins Abkürzungsverzeichnis aufgenommen werden
% Hinweis: allgemein bekannte IT-Begriffe wie Datenbank oder Programmiersprache müssen nicht erläutert werden,
%          aber ggfs. Fachbegriffe aus der Domäne des Prüflings (z.B. Versicherung)

% Die Option (in den eckigen Klammern) enthält das längste Label oder
% einen Platzhalter der die Breite der linken Spalte bestimmt.
\begin{acronym}[WWWWW]
	\acro{SSRS}{Microsoft SQL Server Reporting Services}
	\acro{KVWL}{\betriebName}
	\acro{CFT}{cross-funktionales Team}
	\acro{DWH}{Data Warehouse}
	\acro{VOM}{Ver\-ord\-nungs\-management}
	\acro{BI}{Business Intelligence}
	\acro{KP}{Key Performance}
	\acro{AD}{Active Directory}
	\acro{DI}{Dependency Injection}
	\acro{AOP}{Aspect-Oriented Programming}
	\acro{JPA}{Java Persistence API}
	\acro{ORM}{Object-Relational Mapping}
	\acro{JDBC}{Java Database Connectivity}
	\acro{IDE}{In\-te\-gra\-ted Development Environment}
	\acro{FH}{Fachhochschule}
	\acro{MVC}{Model View Controller}
	\acro{GUI}{Graphical User Interface}
	\acro{POM}{Project Object Model}
	\acro{POJO}{Plain Old Java Object}
	\acro{DTO}{Data Transfer Object}
\end{acronym}
